\startfirstchapter{Introduction}
\label{chapter:introduction}

The development of a personalized learning system began with the creation of an intelligent tutoring system (ITS) \cite{brusilovsky2003adaptive,koedinger1997intelligent,vanlehn2005andes,woolf2010building}. However, ITSs are primarily rules-based which need domain experts to manually specify every possibility that a system might face so it could present appropriate learning actions. This presented a massive combinatorial, labor-intensive challenge as every possible learning path would have to be explicitly specified \cite{lan2016contextual}.\par 

In recent years, machine learning has shown that it has the potential to personalize learning and scale for several courses and students. Machine learning based systems use data to personalize learning actions for each student without the need to explicitly specify learning actions for each individual student. Example of actions could be reading a chapter from a book or article, listening to a podcast, watching a video, or interacting with the system by answering quizzes. These systems are continuously learning from the data generated through students interactions with the system. Thus it has the potential to eliminate the challenges one would face with a traditional ITS  \cite{lan2016contextual}. \par 

\textbf{The goal of this project is to design a learning algorithm, which could adapt based on student's feedback to help them learn effectively}. \par 

\section{Use Case \label{chap1:useCase}}

There is no universal best way to explain a topic. The best way is subjective to every student. Unless we explore different ways to teach a topic, we cannot find a policy which would help map different students to explanations conducive for them. Once, we have such a policy we can use it to teach every student effectively. \textbf{This is the exploration-exploitation dilemma where there is a trade-off between exploration (exploring non-stationarity in a student's preferences) and exploitation (maximize a student's satisfaction over a period of time)} \cite{agarwal2009online}. For example, an adaptive teaching system should present different explanations knowing a student's preference for learning. However, unless we try different ways of teaching it is not possible to say with certainty whether or not an explanation would help a student learn effectively. We use the term adaptive teaching to avoid confusing it with adaptive learning used in machine learning literature. In the education domain, these terms are used interchangeably. \par 

We represent this use case as a contextual bandit problem. We use contextual information about the student such as their preferences to learn through \textit{visual, text, demo-based, practical, activity-based, step-by-step, lecture, audio-based explanations as well as self-evaluation and pre-assessment of students}. We also use contextual information about the content's used to teach a topic, by rating them in terms of \textit{ease of understanding, simplicity, intuitiveness, depth in teaching, conciseness, thoroughness, ratings, abstractness, hands-on, experimental}. A \textbf{content item or arms are different actions or ways a topic can be taught}. The reward would be the student's feedback to confirm their understanding of the topic they are trying to learn. The feedback can be through quizzes, interactions with a content item, tasks to name a few. \textbf{By pulling an arm, we obtain a reward drawn from some unknown distribution determined by the selected content item and the context. Our goal is to maximize the total cumulative reward}.  \par 

Let us make this more concrete by mapping this use case to teaching a class. In any school, a course comprises of multiple topics. However now instead of a single way to teach everyone, there would be multiple ways to teach. These different ways to teach are called content items. Student's give their feedback on the presented content. Behind the scenes, our learning algorithm takes information about the student \textit{(also referred to as student context)}, topic, content items\textit{(also referred to as content context)} to find the best way to teach a student. This project extends the most cited contextual bandit learning algorithm, LinUCB (Linear Upper Confidence Bound) \cite{li2010contextual} to enhance it for our use case. \par

\section{Motivation \label{chap1:motivation}}

The main problem with traditional education, which has been perpetual, is the enormous challenge teachers face for being responsible to ensure every student is able to acquire expertize in their subject even though students may come from diverse backgrounds and interest \cite{educauseReview}. In such classrooms learning has largely remained a one-size-fits-all experience in which the teacher selects a learning resource for all students in their class regardless of their diversity in needs, understanding, ability, preferred learning style, and prior knowledge. It is not feasible for teachers to ensure their explanations can cater to all students. Hence there is a need for a system which could personalize teaching for students to help them learn effectively as well as increase course engagement and progression.\par

Such systems would be adaptive, recognize different levels of prior knowledge among students, as well as course progression based on a student's skill and feedback from learning. This could change teachers responsibility from a provider to a remediator and facilitator in teaching. These would adapt to individual student's learning patterns instead of a student having to adjust to the way of teaching. They would provide timely and comprehensive data-driven feedback to recognize potential challenges that students might come across as the course progresses.\par

\section{Contribution}

We present a novel baseline algorithm for our proposed adaptive teaching methodology which learns from students and contents for each topic to create a personalized learning path for every student. It adapts dynamically based on student's feedback and learning preferences.  \par

We also provide a skip feature which is meant to keep a student engaged to increase their retention as well as provide feedback to teachers by recognizing the challenges faced by a student early in the course. Our online learning algorithm gives close to optimal results over a synthesized unbiased heterogeneous dataset. \par

\newpage
\section{Organization}

% \begin{description}
Chapter 1 provided a brief overview of our use case along with the need for an adaptive teaching system and how this project contributes to realizing it. Chapter 2 introduces the technical concepts used to represent our use case along with the algorithm we customize for adaptive teaching. Chapter 3 describes prior work related to our use case using different approaches and how our work compares to them. Chapter 4 explains the algorithm created for adaptive teaching along with the skip feature. Chapter 5 describes the experimental setup along with the dataset synthesized to evaluate our algorithm. It also explains the evaluation strategy followed to examine our results. Chapter 6 presents the results of our experiments and compares our learning algorithm with respect to the best possible policy. Chapter 7 concludes this project by summarizing the contributions and outlines possible avenues for future work.
% \end{description}