\startchapter{Conclusions}
\label{concl}

This project presents a student-centric approach to teaching. An approach which could make a classroom more interactive by providing a personalized learning experience for students. We synthesized an unbiased dataset to represent heterogeneous student and content data to evaluate our learning algorithm. Since there is no benchmark available, we created an omniscient policy which has optimal parameters pre-configured. The algorithm learns these parameters to find an optimal content item for each student. \par 

We then present a feature which would be useful when there are several different content items for a topic to avoid a student from getting frustrated by being unable to understand a topic. This not only helps students but also helps teachers recognize topics students are less likely to understand. We evaluated the learning algorithm to set a baseline for this new teaching methodology. \par 

Our future work would involve creating an actual course that follows the teaching methods outlined in this project. This would give real-world student data to evaluate the algorithm. We would also like to design other algorithms to evaluate their performance against our baseline algorithm. An additional optimization would be to find an optimal strategy to introduce skipping such that it does not restrict exploration and still provides a good student experience. \par